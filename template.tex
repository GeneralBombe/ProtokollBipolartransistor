%%%%%%%%%%%%%%%%%%%%%%%%%%%%%%%%%%%%%%%%%
% University/School Laboratory Report
% LaTeX Template
% Version 4.0 (March 21, 2022)
%
% This template originates from:
% https://www.LaTeXTemplates.com
%
% Authors:
% Vel (vel@latextemplates.com)
% Linux and Unix Users Group at Virginia Tech Wiki
%
% License:
% CC BY-NC-SA 4.0 (https://creativecommons.org/licenses/by-nc-sa/4.0/)
%
%%%%%%%%%%%%%%%%%%%%%%%%%%%%%%%%%%%%%%%%%

%----------------------------------------------------------------------------------------
%	PACKAGES AND DOCUMENT CONFIGURATIONS
%----------------------------------------------------------------------------------------
\documentclass[
	a4paper, % Paper size, specify a4paper (A4) or letterpaper (US letter)
	12pt, % Default font size, specify 10pt, 11pt or 12pt
]{CSUniSchoolLabReport}
\usepackage[ngerman]{babel}
\usepackage{siunitx}  
\usepackage{textcase}
\usepackage{booktabs}
\usepackage[utf8]{inputenc}
\usepackage{amsmath}
\usepackage{array}
\usepackage{enumitem} % für präzise Kontrolle über Listen
\usepackage{mhchem}
\sisetup{locale = DE,  
separate-uncertainty,  
range-units = brackets,  
list-units = single,  
per-mode=symbol-or-fraction,
}  
\addbibresource{sample.bib} % Bibliography file (located in the same folder as the template)
\renewcommand{\thesection}{\Alph{section}}
\renewcommand{\thesubsection}{\thesection.\arabic{subsection}}
\renewcommand{\thesubsubsection}{\thesubsection.\arabic{subsubsection}}
\makeatletter
\renewcommand\paragraph{\@startsection{paragraph}{4}{\z@}%
  {-3.25ex \@plus -1ex \@minus -0.2ex}%
  {1.5ex \@plus 0.2ex}%
  {\normalfont\normalsize\bfseries}}
\makeatother
\renewcommand{\theparagraph}{\alph{paragraph})}
\newcommand{\micro}{\ensuremath{\mu}}
\newcommand{\pico}{p}
\newcommand{\milli}{m}
\newcommand{\nano}{n}
\newcommand{\RNum}[1]{\uppercase\expandafter{\romannumeral #1\relax}}%----------------------------------------------------------------------------------------
%	REPORT INFORMATION
%----------------------------------------------------------------------------------------
\setlength{\parindent}{0pt}

\title{Protokoll Praktikum EBau Bipolartransistor} % Report title

\author{Johann \textsc{Becker} \and Valentin \textsc{Eder} \and Marc \textsc{Ostner}}
\date{\today} % Date of the report

%----------------------------------------------------------------------------------------

\begin{document}

\maketitle % Insert the title, author and date using the information specified above

\begin{center}
	\begin{tabular}{l r}
		Date Performed: & 30. Mai 2025 \\ % Date the experiment was performed
		
		Instructor: & Prof. Dr. Alexandru Negut % Instructor/supervisor
	\end{tabular}
\end{center}

% If you need to include an abstract, uncomment the lines below
%\begin{abstract}
%	Abstract text
%\end{abstract}

%----------------------------------------------------------------------------------------
%	OBJECTIVE
%----------------------------------------------------------------------------------------

\section{Einführung}
\subsection{Gegenstand des Versuchs}
In diesem Versuch sollen Eigenschaften und Anwendungen des Bipolartransistors BD137-16 untersucht werden.
\subsection{Notwendige Vorbereitungen}
\subsubsection{Versuchsablauf}
Die dynamische Messung der Transistorkennlinien erfolgt ähnlich zu der Messung von Diodenkennlinien.
\subsubsection{Datenblatt}
Der BD137-16 ist ein NPN Silizium Transistor. 
Der "-16" Anhang steht für die dynamische Stromverstärkung $\beta_{\RNum{3}}$, in diesem Fall 100\textasciitilde 250. 

\subsection{Fragen zum Verstärker}

\paragraph{a) Welche Aufgabe hat der Kondensator $C_k$ und wie herum muss ein gepolter Elektrolytkondensator an dieser Stelle eingebaut werden?}
Der Koppelkondensator $C_k$ trennt den Gleichspannungsanteil vom Signal und lässt nur das Wechselspannungssignal durch.
Hierdurch kann ein Transistor Arbeitspunkt unabhängig von der Signalquelle $U_{sig}$ eingestellt werden. Als resultat bleibt der Großsignal Arbeitspunkt bestehen, während die Kleinsignaländerungen von $U_{sig}$ weiterhin bestehen bleiben. 



Ein gepolter Elektrolytkondensator muss so eingebaut werden, dass die positive Seite an die höhere Gleichspannung angeschlossen wird.

















%----------------------------------------------------------------------------------------
%	BIBLIOGRAPHY
%----------------------------------------------------------------------------------------

\section{Formelzeichen}
\begin{table}[h]
\centering
\begin{tabular}{ll}
\toprule
\textbf{Symbol} & \textbf{Bedeutung} \\
\midrule
$C_s$    & Sperrschichtkapazität \\
$I_c$    & Fluß- oder Vorwärtestrom \\
$I_a$    & Sperrstrom- oder Rückwärtestrom \\
$I_s$    & Sperrsättigungsstrom \\
$M$      & Stufenfaktor (grading coefficient) \\
$m$      & Emissionskoeffizient \\
$N_a$    & Akzeptordichte \\
$N_0$    & Donatordichte \\
$R_a$    & Bahnwiderstand \\
$t_0$    & Injektionszeit \\
$t_1$    & Anstiegszeit (Risetime) \\
$t_1$    & Sperrverzögerungszeit (Reverse Recovery Time) \\
$t_s$    & Speicherzeit \\
$U_0$    & Diffusionsspannung \\
$U_c$    & Fluß- oder Vorwärtsspannung \\
$U_s$    & Sperr- oder Rückwärtsspannung \\
$U_{s\_5mk}$ & Durchbruchspannung an der Z-Diode bei $I_2 = 5$mA \\
\bottomrule
\end{tabular}
\end{table}

\printbibliography % Output the bibliography

%----------------------------------------------------------------------------------------

\end{document}